\section{Conclusion}
Through the analysis of the Apple Watch it was found that three sensors
potentially could be used for biometric sampling of user characteristics. 
The three sensors being the heart rate sensor, the microphone and the motion
sensors. The limitations of the platform were identified and with these and the
available sensors taken into account, three possible systems were proposed.
Upon further exploration of the system, it was found that the limitations
hindered the development of a complete biometric system for identification of
subjects. With this in mind a prototype for sensor data extraction was
developed. The prototype was documented with code listings, covering the usage
of Apples frameworks for data extraction. This documentation can be utilized as
a reference in further development of prototypes on the platform. It was found
that the watch has limited access to raw data from the heart rate sensor, and
that some usability features, like the automatic screen control and not being
able to run background services, hinders the watch in being a capable biometric 
identification device. With this said, future revisions could open up the
platform, allowing developers greater control of sensors and behaviour.
 
