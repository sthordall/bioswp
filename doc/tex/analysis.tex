\section{Analysis}
This section will analyze the potential for capturing biometrics on the
\textit{Apple Watch}. The analysis will cover both the hardware and
software, i.e.\ which sensors are included in the watch and how they are 
utilized by the provided frameworks. The sensors will also be evaluated 
on their usefulness for biometric identification. Finally the analysis 
should result in a selection of sensors found appropriate for the prototype.

\subsection{Sensors}
The \textit{Apple Watch} includes a multitude of sensors \cite{watchspecs}, 
which are used mainly for usability, activity, fitness and health tracking, but
they are also used for some security functionalities, such as using the heart 
rate sensor, to ensure the owner who unlocked the watch, has not removed the 
watch since authorization occurred.

\subsubsection{Heart rate sensor}
The integrated heart rate sensor uses the technology
\textit{photoplethysmography}. Which functions by using green LED's to
illuminate the veins in the wrist and photodiodes to detect the amount of blood
flowing in said veins. This allows the watch to detect the wearers heart rate
\cite{watchheartrate}.
The quality of measurements rely on the wearers fit of the watch, but when
fitted correctly heart rate data is provided through Apple's \textit{HealthKit}
framework. The heart rate data can be fetched as a stream through an instance of
\texttt{HKHealthStore}, and can both be fetched as a calculated heart rate, and
as raw sensor data \cite{healthkitfw}.
Heart rate data for biometric identification has been explored by other vendors
than Apple \cite{nymiwhitepaper}, but this has been done with different sensors
and raw sensor data. It has been found possible with raw sensor data to identify
individuals from photoplethysmography signals, like the ones obtained from the
Apple Watch sensor \cite{kavsaoglu2013a}. A photoplethysmography sensor
therefore seems like a potential biometric characteristic capture device.


\subsubsection{Accelerometer \& Gyroscope}
Both an accelerometer and a gyroscope are located in the watch, and these are
used for features like activity tracking and rotation detection for screen
auto on/off. Raw data is accessible from both sensors through the
\textit{CoreMotion} framework, from \texttt{CMAccelerometerData} and
\texttt{CMGyroData} accordingly \cite{coremotionfw}. The accelerometer allows
for 3-axis movement tracking, and the gyroscope detects rotation. These in
combination allow for precise movement tracking, but are limited to track the
movement of the wrist. This has been used for activity tracking and pedometer 
in many smart devices, but one might need to investigate the precision of this
before using it for identification. 
Experiments on identifying individuals from motion data has been tested, and
one approach could be gait recognition. Gait recognition allows for
identification from how the subject walks, and research points towards this
being possible from similar sensors within smartphones \cite{7181946}. 

\subsubsection{Microphone}


\subsubsection{NFC}
The potential of an NFC chip in the watch seems promising for access control, as
it could allow for interaction with third-party devices, i.e.\ NFC readers. This
could be utilized for doors which rely on NFC key cards, where the watch could
function as a secure device containing the NFC key cards. 
Unfortunately this is simply not possible, as Apple has restricted developer 
access to the NFC chip, only allowing it to be used with their own \textit{Apple
    Pay} service. 

\subsection{Limitations}
The SDK does not allow for full control of all aspects of the watch. There are
some limitations to the way Apple runs third-party applications on the platform.
Most of the limitations are of course present due to user experience and battery
lifetime constraints. The watch does not allow for continuous background services, it is
dependent on having an iPhone connected to enable internet connectivity, apps can not
hinder the screen in automatically shutting off and the processing power is of
course limited in comparison to smartphones and laptops.

The main limitations affecting the development of the prototype are the lacking
ability to run background services and the automatic screen on/off.

The ability to run background services would be necessary if something like gait
recognition should function. This is not possible, and limits the continuous
tracking of an individual for identification purposes. 

If identification should rely on movement data from the accelerometer and
gyroscope, the automatic screen control is a limiting functionality. When one
moves the watch the screen very likely turns off, and the process halts. It is
possible to start fitness sessions and extract the data from here ...
% TODO: Include auto screen turn-off, continuous background services, dependency
% on phone, platform independence, limited processing power

\subsection{Possible systems}
% TODO: Mention possible setups, from findings in analysis
With the sensors and limitations taken into account, two possible systems seems
probable. Common to both systems are the users active participation, and the
watching being still. This is to overcome the limitations of automatic screen
on/off and not being able to run background services.

\subsubsection{Voice Recognition}
This approach could utilize the built-in microphone to record and extract
features from the subjects voice. The detection could rely both on the words
being said and not. Voice recognition in general has seen a growth in use, due
to the growing popularity and efficiency of modern machine learning
technologies. This approach would therefore be both possible and convenient for
users, already comfortable with talking to their smart devices. 
The system would need to extract features of the subjects voice, not only said
words. This is quite challenging and might also need machine learning tactics to
perform adequately. Due to the limited processing power of the Apple Watch, and
the requirements from machine learning, the recorded sound samples might need to
be sent to a central server for processing. This could result in privacy issues,
   and should therefore be investigated more thoroughly. % TODO: Reference article

\subsubsection{Photoplethysmography Recognition}

