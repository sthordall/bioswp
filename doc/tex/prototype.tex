\section{Prototype}
% TODO: How does the prototype handle sample quality?
%           - Length of clips?
%           - Performed in quiet environment?
Due to the findings in the analysis section, the prototypes scope has been
limited to feature extraction from the different sensors in the watch. This means
that no feature comparison will be performed. This was found to be a necessary
compromise, as all the proposed systems requires advanced machine learning to
function optimally. An implementation of the \textit{Dynamic Time Warp}
algorithm has also been included, to test if any comparison algorithm is
possible on a smart watch. Much of the code has been inspired by open source
examples \cite{watchosheartratesamplerepo} \cite{healthkitheartrateexporter} 
\cite{watchossampler}, found on the popular code repository site \textit{GitHub}.
The main purpose of the prototype is therefore to sample the sensors, allowing
for further analysis of the output.

The prototype includes feature extraction from the three possible sensors,
    i.e.\ the microphone, accelerometer and heart rate sensor. These should be
easily sampled, and extracted from the device, this process will also be
covered. 

\subsection{User Interface}

\subsection{Microphone}

\subsection{Motion Sensors}

\subsection{Heart Rate Sensor}

\subsection{Dynamic Time Warp}

\subsection{Data transfer and storage}
