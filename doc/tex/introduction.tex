\section{Introduction}
The miniaturization of computing hardware which spawned the smartphone era, has also
allowed for the development of powerful smartwatches filled with sensors.
These sensors can continuously observe the wearers bodily functions, which in 
theory can be utilized for biometric identification of the wearer.
This could allow for a strong link between the user and smartwatch, resulting in
unobtrusive and ubiquitous user authentication.

This paper will analyse the first generation of smartwatches, more specifically
the \textit{Apple Watch}, to find which biometric capabilities that can be
utilized for user identification and authentication. The analysis will look into
the available sensors and the associated frameworks provided by Apple, in order 
to find which biometric measurements can be performed. The paper will try to 
identify both possibilities and limitations of the system, and try to utilize 
the possibilities in a prototype, to see how this in practice could be 
implemented. 

The implemented prototype will be covered in the format of code listings, 
describing how the data is extracted from the sensors, stored on the device and,
if possible, processed by comparison algorithms. The listings will be
supplemented by descriptions of how the associated frameworks provided by 
Apple are utilized.
