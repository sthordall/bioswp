\section{Introduction}
The minimization of computing hardware which spawned the smartphone era, has also
allowed for the development of powerful smartwatches filled with sensors.
These sensors can continuously observe the wearers bodily functions, which in 
theory can be utilized for biometric identification of the wearer.
This could allow for a strong link between the user and smartwatch, resulting in
unobtrusive and ubiquitous user authentication. % TODO: Rewrite last part

This paper will analyse the first generation of smartwatches, more specifically
the \textit{Apple Watch}, to find which biometric capabilities that can be
utilized for user identification and authentication. The analysis will look into
the available sensors and the \textit{Software Development Kit (SDK)} provided 
by Apple, in order to find which biometric observations can be performed. The
paper will try to identify both possibilities and limitations of the system, and
try to utilize the possibilities in a prototype, to see how this in practice
could be implemented. The prototype will include feature extraction from
biometric measurements and a comparison of these, in order to identify the
wearing individuals. Lastly the prototype will be tested on 10 subjects, to
evaluate the prototypes performance.
